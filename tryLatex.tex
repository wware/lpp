\documentclass[11pt]{article}
\title{Literate Programming Preprocessor}
\author{Will Ware \texttt{<wware@alum.mit.edu>}}
\date{}

% PDFLaTeX Support
\newif\ifpdf
\ifx\pdfoutput\undefined
   \pdffalse
\else
   \pdfoutput 1
   \pdftrue
\fi

\begin{document}

\thispagestyle{plain}
\pagenumbering{roman}
\maketitle

\section{Literate programming preprocessor}

Knuth's original vision for literate programming involved a single source file that
could be processed in two different ways to produce compilable code on one pathway
and a TeX/LaTeX source doc on the other. This approach diverges from that, using two
sources, one of them an unmodified source code file.

There are a few reasons for this divergence. First, the large volume of legacy code
that hadn't yet been written in Knuth's day. Second, the large number of very useful
tools that now exist and that assume their input is a normal source code file, not
some squirrely predecessor to a source code file. Third, modern engineers don't want
to learn some new scheme of deriving their source code from some other document
format.

So the first source for this scheme is a normal source code file, with no modifications
whatsoever on behalf of this approach. There are no extra comments or tags or markup
of any sort. Zero impact on source code is a hard requirement of this approach.

Also, the preprocessor should be as agnostic as possible about how documents are
processed. LaTeX, Markdown, HTML, and any other format should work with minimal
distraction. Currently there are examples for LaTeX and Markdown.

To include a code snippet into a document, you use a sequence of regular expresssions.
The last two in the sequence specify the starting and ending line of the snippet
and the earlier regexes are steps to get you to the staring line. For instance you
get to a particular method in a particular class in a particular file with something
like this, where the "@" sign is the left-most character in the line, signaling the
preprocessor that this line is to be preprocessed.

\begin{verbatim}
    @foobar.py:class Foobar/def foo\(self\)/x = 3/print x
\end{verbatim}

produces

\begin{verbatim}
@foobar.py:class Foobar/def foo\(self\)/x = 3/print x
\end{verbatim}

To generate a PDF from this LaTeX source, run "make".

\end{document}
